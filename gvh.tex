\documentclass[10pt,a4paper]{article}

\usepackage[brazil]{babel}
\usepackage[utf8x]{inputenc}
\usepackage[T1]{fontenc}

\usepackage[a4paper,top=3cm,bottom=2cm,left=3cm,right=3cm,marginparwidth=1.75cm]{geometry}

\usepackage{amsmath}
\usepackage{graphicx}
\usepackage[colorinlistoftodos]{todonotes}
\usepackage[colorlinks=true, allcolors=blue]{hyperref}

\title{IF684 - Sistemas Inteligentes}
\author{Gabriela Vilela Heimer}

\begin{document}
\maketitle

\begin{figure}[h]
\centering
\includegraphics[width=0.5\textwidth]{AI.jpg}
\caption{\label{fig:AI}CC0 Creative Commons}
\end{figure}

% link da imagem :https://pixabay.com/pt/intelig%C3%AAncia-artificial-ai-rob%C3%B4-2228610/
% licença CC0 Creative Commons

\section{Introdução}

Sistemas Inteligentes é uma disciplina que se insere na área de Inteligência Artificial, o ramo da computação que estuda a automação de comportamento inteligente. No estudo dessa disciplina, os alunos irão aprender a resolver problemas de grande complexidade que não possuem uma resolução algorítmica, encontrando as técnicas mais apropriadas para suas soluções. Os livros base dessa matéria são \cite{AI:NewSynthesis} e \cite{AI:ModernApproach}, e além disso, os sites da cadeira (\cite{SiteCadeira1} e \cite{SiteCadeira2}) mostram diversos outros livros e artigos da área, dando aos alunos muitas opções de aprendizado.

\section{Relevância}

Sistemas Inteligentes é uma disciplina essencial para um graduando de Ciência da Computação, afinal a área da Inteligência Artificial vem crescendo muito na atualidade. O aprendizado dessa disciplina tem muitos pontos positivos:

\begin{enumerate}
\item Essa cadeira da aos alunos uma visão geral sobre a Inteligência Artificial, fazendo com que eles entendam as diversas subáreas desse campo e com que eles possam se especializar futuramente nas subáreas que mais os interessam .
\item Por conta do constante crescimento dos conhecimentos que temos sobre a IA (inteligência artificial), ela tem um papel importantíssimo na área de pesquisa.
\item A Inteligência Artificial é aplicada em diversas áreas da computação, como na Robótica, em jogos, na computação gráfica, entre muitos outros, dando aos futuros formandos um leque de oportunidades.
\item Essa grande área da computação tem um mercado muito lucrativo que,  assim como a área de pesquisa, vem crescendo bastante.
\end{enumerate}

Porém, há um ponto negativo:

\begin{enumerate}
\item Por conta de IA ser uma área muito abrangente, não há tempo nessa cadeira para entrar em detalhes no estudo das sub-áreas.
\end{enumerate}

\section{Relação com outras disciplinas}

\begin{table}[h!]
\centering
\label{my-label}
\begin{tabular}{|l|l|}
\hline
IF672 - Algoritmos e Estruturas de Dados  & \begin{tabular}[c]{@{}l@{}}Essa cadeira se relaciona com Sistemas \\ Inteligentes pois foca na otimização do \\ programa, o que é essencial para a \\ aplicação de IA.\end{tabular}              \\ \hline
IF673 - Lógica para Computação            & \begin{tabular}[c]{@{}l@{}}Essa cadeira aprimora o raciocínio dos \\ estudantes e ajuda no entendimento de \\ conceitos básicos, essenciais para a \\ compreensão de IA.\end{tabular}            \\ \hline
IF793 - Projeto Implementação de Jogos 2D & \begin{tabular}[c]{@{}l@{}}O uso de IA em jogos e simulações é muito \\ frequente, tornando o conhecimento dessa \\ área fundamental para o estudo dessa cadeira.\end{tabular}                   \\ \hline
IF962 - Recuperação de Informação         & \begin{tabular}[c]{@{}l@{}}A Inteligência Artificial é usada na recuperação \\ de informação, logo Sistemas Inteligentes é uma \\ cadeira necessária para o estudo dessa\\ matéria.\end{tabular} \\ \hline
\end{tabular}
\end{table}

\nocite{CInWIKI}
\bibliographystyle{plain}
\bibliography{sample}

\end{document}