\documentclass[a4paper,10]{article}

%% Language and font encodings
\usepackage[brazil]{babel}
\usepackage[utf8x]{inputenc}
\usepackage[T1]{fontenc}

%% Sets page size and margins
\usepackage[a4paper,top=3cm,bottom=2cm,left=3cm,right=3cm,marginparwidth=1.75cm]{geometry}

%% Useful packages
\usepackage{amsmath}
\usepackage{graphicx}
\usepackage[colorinlistoftodos]{todonotes}
\usepackage[colorlinks=true, allcolors=blue]{hyperref}

\title{IF674 - Infraestrutura de hardware}
\author{Samuel Oliveira de Miranda}

\begin{document}
\maketitle



\section{Introdução}

Em termo gerais, infraestrutura de hardware é a área que estudará a estrutura e o comportamento funcional das máquinas. Nela o estudante terá uma visão mais amplas sobre os mais variados componentes de um computador, além de compreender todo o processo lógico por trás de cada um. 
\\
\textit{ "Um computador pode ser visto como um sistema formado por um conjunto estruturado de
componentes, e sua função pode ser compreendida em termos das funções desses componentes."
\cite{Stalling}}

\begin{figure}[h]
\centering
\includegraphics[width=0.3\textwidth]{Hardware.png}
\caption{\label{fig:PC}Alguns principais dispositivos de hardware.}
\end{figure}

%Link da figura 1:
%https://commons.wikimedia.org/wiki/File:Personal_computer,_exploded.svg
%Licença da imagem:
%https://creativecommons.org/licenses/by-sa/3.0/deed.en
%(CC BY-SA 3.0)

\section{Relevância para o curso}

Em Infraestrutura de Hardware o aluno não se restringirá à apenas conteúdos abstratos, mas também concretizará os conhecimentos adquiridos ao longo das aulas em um projeto no qual o aluno deverá criar uma versão básica de um desses componentes. A disciplina prepara os alunos para especificar, adquirir e orientar na manutenção de equipamentos de hardware, seja abrangendo desde a definição de computadores para utilizar em uma determinada tarefa, até para atividades mais complexas como projetos de máquinas.  \href{http://www.cin.ufpe.br/~if674/Objetivo.html}{[2]} 

\begin{figure}[h]
\includegraphics[width=0.1\textwidth]{download.png}
\centering
\caption{\label{fig:PC}Ilustração de um microchip.}
\end{figure}

%Link da figura 2:
%https://pixabay.com/pt/chip-%C3%ADcone-micro-processador-1710300/
%Licença da imagem:
%https://creativecommons.org/publicdomain/zero/1.0/deed.pt
%CC0 1.0 Universal (CC0 1.0)

\subsection{Pontos Positivos de se estudar a disciplina}

É fundamental para um profissional da área de computação ter ao menos conhecimentos básicos de arquitetura de computadores, saber o seu funcionamento é essencial para compreendermos como os componentes se interligam e formam sistemas complexos, bem como melhor entender, analisar e melhorar a sua principal ferramenta de trabalho.  

\newpage
\section{Relação com outras disciplinas}

Na computação há uma grande rede de correlação, onde as mais variadas áreas acabam se interligando e relacionando-se entre si, em infraestrutura de hardware algumas dessas disciplinas são mostradas na tabela a seguir:


\begin{table}[!h]
\centering

\label{my-label}
\begin{tabular}{|l|l|}
\hline
Disciplinas relacionadas:           & Descrição da a disciplina:                                                                                                                                                                                                                           \\\hline
IF678-Infraestrutura de Comunicação & \begin{tabular}[c]{@{}l@{}}Nela o aluno passará a entender os diversos aspectos\\ de projetos e implementação de redes de computadores, além de\\ como funcionam a internet e os diversos protocolos de comunicação \\existentes.\href{https://www.cin.ufpe.br/~pet/wiki/Infraestrutura_de_Comunica%C3%A7%C3%A3o}{[IF678]}
\end{tabular}                           \\\hline
IF677-Infraestrutura de Software    & \begin{tabular}[c]{@{}l@{}}Esse curso dará ao aluno entendimento necessário para  saber o\\ funcionamento dos sistemas de software que fornecem uma\\ infraestrutura através da qual  aplicativos podem interagir\\ com o hardware.\href{https://sites.google.com/a/cin.ufpe.br/if677/}{[IF677]}\end{tabular}                 \\\hline
IF675-Sistemas Digitais             & \begin{tabular}[c]{@{}l@{}}Esse curso dará ao aluno conhecimentos de circuitos lógicos digitais\\ cobrindo desde dispositivos digitais de pequena complexidade até\\ a implementação circuitos de média complexidade.\href{https://www.cin.ufpe.br/~pet/wiki/Sistemas_Digitais}{[IF675]}

\end{tabular}
\\\hline
\end{tabular}
\caption{Disciplinas relacionadas.}
\end{table}

\bibliographystyle{plain}
\bibliography{som3}

\end{document}