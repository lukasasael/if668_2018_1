\documentclass[a4paper]{article}

%% Language and font encodings
\usepackage[portuguese]{babel}
\selectlanguage{portuguese}
\usepackage[utf8x]{inputenc}
\usepackage[T1]{fontenc}
\fontsize{10}{10}

%% Sets page size and margins
\usepackage[a4paper,top=3cm,bottom=2cm,left=3cm,right=3cm,marginparwidth=1.75cm]{geometry}

%% Useful packages
\usepackage{amsmath}
\usepackage{graphicx}
\usepackage[colorinlistoftodos]{todonotes}
\usepackage[colorlinks=true, allcolors=blue]{hyperref}

\title{IF754 - Computação musical e processamento de som}
\author{Vinícius Padilha Garcia}

\begin{document}
\maketitle

\section{Introdução}

	A disciplina de Computação musical e processamento de som trata desde os fundamentos do que é som, acústica e música até os aspectos técnicos de fenômenos computacionais acerca disso, como o processo de síntese sonora ou a recuperação de informação sonora. Para isso, as aulas são divididas entre teóricas e práticas e contam com uma dupla de professores reconhecidos por sua atuação na área: Geber Ramalho e Giordano Cabral. O estudo da computação musical pode ser colocado dentro do estudo da multimídia, e o viés dado às aulas durante o período depende, segundo o professor Giordano, muito do interesse dos estudantes, o que permite que abranjam diferentes assuntos, como a criação de interfaces controladoras digitais, ou "instrumentos sintetizadores", e a utilização da arquitetura MIDI, por exemplo.

\section{Relevância}

	A aplicação de qualquer estudo científico tecnológico no meio artístico é de grande valor, ou, no mínimo, interessante. Isso tanto por dissolver a barreira existente e promover interação entre áreas com abordagens tão diferentes quanto pelo desenvolvimento da cultura humana, que não será algo diferente de digital na era da informação que hoje observamos nascer. Sendo assim, a possibilidade de estudar essa disciplina é propriamente valiosa tanto para os estudantes da universidade quanto para a academia e até para a evolução da música eletrônica.

\subsection{Pontos positivos e negativos}

Positivos

\begin{itemize}
\item Interação com o meio artístico;
\item Professores ativos na área;
\item Flexibilidade de conteúdo;
\item Aulas teóricas e práticas.
\end{itemize}

Negativos

\begin{itemize}
\item Relatos de "aulas teóricas chatas";
\item Site da disciplina desatualizado.
\end{itemize}

\begin{figure}
\centering
\includegraphics[width=0.8\textwidth]{DiferentesFormatosDeOnda.jpg}
\caption{\label{fig:DiferentesFormatosDeOnda}Diferentes formatos de onda representados em software de produção musical.}
\end{figure}
\pagebreak
\section{Relação com outras disciplinas}

\begin{table}[! th]
\centering
\begin{tabular}{l|r}
Código/disciplina & Relação com IF754\\\hline
IF681 - Introdução a multimídia & trabalho com mídia sonora\\
IF687 - Interfaces usuário-máquina & controladores MIDI (criação de instrumentos)\\
IF793 - Projeto implementação de jogos 2D & possibilidade de trabalho com trilhas sonoras de jogos
\end{tabular}
\caption{\label{tab:widgets}Representação tabular de relações entre IF754 e outras disciplinas.}
\end{table}

\section{Referências bibliográficas da disciplina}

Três referências bibliográficas da disciplina agora serão referenciadas.
\begin{itemize}
\item \cite{roads1996computer};
\item \cite{moorer1977signal};
\item \cite{mathews1969technology}.
\end{itemize}

\bibliographystyle{unsrt}
\bibliography{referencias}

\end{document}