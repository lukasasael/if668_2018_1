\documentclass[a4paper]{article}

%% Language and font encodings
\usepackage[brazil]{babel}
\usepackage[utf8x]{inputenc}
\usepackage[T1]{fontenc}
\fontsize{10}{10}\selectfont

%% Sets page size and margins
\usepackage[a4paper,top=3cm,bottom=2cm,left=3cm,right=3cm,marginparwidth=1.75cm]{geometry}

%% Useful packages
\usepackage{amsmath}
\usepackage{graphicx}
\usepackage[colorinlistoftodos]{todonotes}
\usepackage[colorlinks=true, allcolors=blue]{hyperref}
\usepackage{indentfirst}

\title{IF682 - Engenharia de Software e Sistemas}
\author{Douglas Ventura}

\begin{document}
\maketitle
\section{Introdução}

A cadeira de \textit{Engenharia de Software e Sistemas} fornece ao estudante ferramentas e técnicas para entender e solucionar o desenvolvimento de diferentes tipos de sistemas de software. Oferecendo também uma visão geral, de como construir um projeto, de como se relacionar e trabalhar em equipe, incluindo também a parte ética de como se comportar e reconhecer limitações em projetos.

\section{Relevância}

De acordo com o Ian Sommerville, referência no mundo da computação e autor do livro-texto\cite{sommerville2011software} da disciplina, diz que "O mundo moderno simplesmente não existiria sem o software". Essa afirmação torna-se válida pelo fato de vivermos em um mundo altamente conectado, onde tudo é dependente de computadores onde os sistemas são a base para tudo funcionar. E enteder os processos da construção de um software é impressindível para um estudante de computação.

A área é rica pela diversidade de desenvolver distintos tipos de sistemas, para diferentes fins. Passando por várias partes, desde como projetar e construir uma solução e até entender os processos na criação de um software. A página da disciplina no \href{https://www.cin.ufpe.br/~pet/wiki/Engenharia_de_Software_e_Sistemas}{CinWiki} mostra que a cadeira vai desde a decisão do projeto, até a evolução do sistema, passando pelas fases de implematação, desenvolvimento, testes e manuntenção. Visando organização, produtividade e qualidade. 

\begin{figure}[h]
\centering
\includegraphics[width=0.60\textwidth]{imagem-01.jpeg}
\caption{\label{fig:imagem-01}Exemplo de etapas na construção de um Software.}
%% Licença da Imagem: https://commons.wikimedia.org/wiki/File:SDLC_-_Software_Development_Life_Cycle.jpg
%% Autor: Cliffydcw
%% Tipo da Licença: CC BY-SA 3.0
%% Editei a imagem e coloquei o nome do autor e o link.
\end{figure}

\pagebreak

O aluno que cursa a disciplina é desafiado com projetos em equipe para desenvolver as habilidades e a importância de trabalhar coletivamente como abordado no livro \cite{mt2010desenvolvimento}. A cadeira também trata as questões éticas, de como respeitar a confidenciabilidade de seus clientes ou funcionários, de modo honesto e responsável para ser respeitado como profissional\cite{slidedaaula}.

\subsection{Pontos positivos}
\begin{itemize}

\item Incrementa ao aluno habilidades no desenvolvimento prático de um software com qualidade.
\item Fornece interação social e como trabalhar em equipe no desenvolvimento de projetos.
\item Incentiva em questões éticas de relacionamento que vão além do conhecimento técnico.

\end{itemize}


\subsection{Pontos Negativos}
\begin{itemize}

\item Pouco tempo para estudar uma área tão vasta.

\end{itemize}

\section{Relação com outras disciplinas}

\begin{table} [h]
\centering
\begin{tabular}{|l|p{9.0cm}|} \hline

Disciplina & Relação \\\hline

IF672 - Algoritmos e Estruturas de Dados & A cadeira fornece técnicas de como construir algoritmos mais complexos de forma a otimizar a resolução de problema. Tornando posteriormente essa cadeira essencial para Engenharia de Software e Sistemas. \\\hline

IF673 - Lógica para Programação & Fornece a base teórica da matemática lógica, ampliando o raciocínio do aluno para implementações em futuros sistemas complexos. \\\hline

\end{tabular}
\end{table}

\bibliographystyle{unsrt}
\bibliography{dvsf}

\end{document}