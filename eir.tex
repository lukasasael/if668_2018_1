 \documentclass[a4paper]{article}

%% Language and font encodings
\usepackage[brazil]{babel}
\usepackage[utf8x]{inputenc}
\usepackage[T1]{fontenc}

%% Sets page size and margins
\usepackage[a4paper,top=3cm,bottom=2cm,left=3cm,right=3cm,marginparwidth=1.75cm]{geometry}

%% Useful packages
\usepackage{amsmath}
\usepackage{graphicx}
\usepackage{float}
\usepackage[colorinlistoftodos]{todonotes}
\usepackage[colorlinks=true, allcolors=blue]{hyperref}
 \title{IF675 - Sistemas Digitais }
\author{Eduardo Inácio Rodrigues}
\begin{document}
\maketitle
\begin{abstract}
A cadeira de Sistemas Digitais, lecionada pelo Professor Manoel Eusebio de Lima no CIn-UFPE, intenta ensinar ao aluno sistemas lógicos digitais. Durante o seu estudo, os alunos produzem projetos em CAD(Computer Aided Design), com o objetivo de conhecer dispositivos que integram processadores de informação em nosso cotidiano. 
\end{abstract}
\section*{Introdução}
Nessa cadeira são estudados os princípios básicos de sistemas embarcados
pelos pontos de vistas teórico e prático, perpassando pelo conhecimento das lógicas
combinacionais e sequenciais de complexidade SSI(Small Scale Integration) até MSI(Medium Scale Integration). Além disso, é importante
para o discente aprender a desenvolver o fluxo de projeto utilizando-se de
ferramentas de CAD, a fim de desenvolver tanto a análise quanto a montagem. Dentre os tópicos oferecidos pelo
curso está o conhecimento, a simplificação e a especificação de variáveis e de funções
lógicas, além de teoremas e operações aritméticas. Sistemas Embarcados são sistemas que apresentam apenas uma função, e tem integração com o meio ambiente. A exemplos de sistemas embarcados, temos: carros, aviões, etc.
\begin{figure} [H]
\centering
\includegraphics[width=0.3\textwidth]{325px-Half-adder_svg.png} 
\caption{\label{fig:frog}Exemplo de Sistema Digital}
\end{figure}

\section*{Relevância}
Em virtude da complexidade e presença dos circuitos digitais no dia-
a-dia, estando presente na maioria dos aparatos tecnológicos e da projeção desse
aspecto para o futuro, é de suma importância o estudo da disciplina. Ao fim do
período, os alunos devem estar aptos a criar e implementar qualquer circuito
combinatório ou sequencial dos campos SSI E MSI. Outrossim, devem estar prontos
para compreender todo o funcionamento desses circuitos; conhecer as
particularidades da lógica digital; relacionar problemas de lógica; relacionar circuitos
integrados e poder fazer simplificações.
\linebreak

\section*{ Relação com outras disciplinas}
\begin{table}[H]
\centering
\label{my-label}
\begin{tabular}{|l|l|}
\hline
IF674-Infra-estrutura de Hardware & \begin{tabular}[c]{@{}l@{}}Visando o estudo de circuitos sequenciais e combinacionais, é de extrema\\ importância o estudo de Sistemas Digitais para o aprendizado de\\ Infra-estrutura de Hardware\end{tabular}                                        \\ \hline
IF677-Infra-estrutura de Software & \begin{tabular}[c]{@{}l@{}}É necessário o estudo de Sistemas Digitais para o aprendizado de\\ Infra-estrutura de Software, pois é ministrado em Sistemas digitais\\ a simplificação e a especificação de variáveis e de funções\\ lógicas.\end{tabular} \\ \hline
\end{tabular}
\end{table}


\bibliographystyle{alpha}
\bibliography{eir}

\cite{ercegovac2000introduccao}
\cite{booth1984introduction}
\cite{taub1984circuitos}\\
\linebreak
\url{https://commons.wikimedia.org/wiki/File:Half-adder.svg}
\end{document}