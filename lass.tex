\documentclass[10pt,a4paper] {article}
\usepackage[utf8]{inputenc}
\usepackage[T1]{fontenc}


\usepackage{fullpage}
\usepackage{newtxtext,newtxmath}
\usepackage{graphicx}
\usepackage{listings}
\lstset{language={[LaTeX]TeX},basicstyle=\ttfamily}

\usepackage{subfiles}

\title{IF769 - Mineração na Web}
\author{Lucas Antonio Sales dos Santos}
\usepackage[brazil]{babel}
\begin{document}
	\maketitle	
	\centering
	\section{Introdução}
	\centering
A disciplina de Mineração na Web atua na área de Big Data, tem como foco a recuperação de informação, a disciplina possui 3 módulos, onde no primeiro módulo tem o conteúdo focado em Recuperação de informação(Clássica), onde são introduzidos a Recuperação de informação pelo método de indexação e busca de documentos e ainda como conteúdo, a avaliação de sistemas de recuperação de informação. No segundo módulo os conteúdos são Recuperação inteligente de informação, mineração de texto, extração de informação e etc. E no terceiro módulo são feitos projetos de mineração na web pelos alunos. 

	\section{Relevância}
	\centering
A disciplina é de extrema importância, pois a Web é hoje a maior fonte de informação eletrônica que temos, mas por não termos controle sobre ela, achar informações relevantes pode se tornar algo que em muitas vezes é frustrante,e muitos esforços de pesquisa têm sido feitos para sanar esse problema. E um deles é a utilização de técnicas de mineração de dados para a descoberta de informações na web. Mas, utilizar e compreender os dados disponíveis na Web não é tarefa fácil, pois esses dados são muito mais sofisticados e dinâmicos do que os sistemas de armazenamento de banco de dados tradicionais. Enquanto estes utilizam de estruturas de armazenamento bem definidas e estruturadas, a Web não possui qualquer controle sobre a estrutura ou o tipo dos documentos que virtualmente armazena, por isso é muito importante que esta disciplina seja ofertada.
\\ 

	\section{Relação com outras disciplinas}	
	
	

	\centering 
	
	\begin{table}[]
		\centering
		
		\label{my-label}
		\resizebox{\textwidth}{!}{%
			\begin{tabular}{|c|c|}
				\hline
				Disciplina & Relação \\ \hline
				IF695 -BANCO DE DADOS AVANÇADOS+ ELETIVO & Relação direta pois o banco de dados armazena os dados e esses dados serão minerados para utilização. \\ \hline
				IF694 -BANCO DE DADOS DISTRIB. E MOVEIS+ ELETIVO & Relação direta pois a mineração de dados vai acessar os bancos de dados. \\ \hline
				IF693 -SIST.GERENC. DE BANCO DE DADOS+ ELETIVO & Relação direta pois a mineração de dados vai utilizar dados armazenados por esses bancos de dados. \\ \hline
				IF697 -TOP.AVANC.GERENC.DADOS INFORMACAO+ ELETIVO & Banco de dados e mineração estão diretamente relacionados. \\ \hline
				IF696 -INTEG. DADOS WEB E WAREHOUSE+ ELETIVO & Banco de dados e mineração estão diretamente relacionados. \\ \hline
				IF692 -PROJETO DE BANCO DE DADOS+ ELETIVO & Banco de dados e mineração estão diretamente relacionados. \\ \hline
				IF698 -SEMIN. GERENC. DADOS E INFORMACAO+ ELETIVO & Banco de dados e mineração estão diretamente relacionados. \\ \hline
				IF685 -GERENCIAMENTO DADOS E INFORMACAO OBRIGATÓRIO & Banco de dados e mineração estão diretamente relacionados. \\ \hline
			\end{tabular}%
		}
	\end{table}
	\clearpage		
\newpage



	\section{Referências}
	\begin{quotation}
\centering
 http://trec.nist.gov/ - The TREC (Text REtrieval Conference) series is co-sponsored by the NIST 
 http://infolab.stanford.edu/~backrub/google.html\\
 https://www.nist.gov/itl\\
 http://www.cs.utexas.edu/users/mooney/ir-course/\\
	\end{quotation}


\end{document}
