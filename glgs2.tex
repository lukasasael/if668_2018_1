\documentclass[a4paper]{article}

%% Language and font encodings
\usepackage[brazil]{babel}
\usepackage[utf8x]{inputenc}
\usepackage[T1]{fontenc}

%% Sets page size and margins
\usepackage[a4paper,top=3cm,bottom=2cm,left=3cm,right=3cm,marginparwidth=1.75cm]{geometry}

%% Useful packages
\usepackage{amsmath}
\usepackage{graphicx}
\usepackage[colorinlistoftodos]{todonotes}
\usepackage[colorlinks=true, allcolors=blue]{hyperref}

\title{IF681 - Interface Usuário Máquina}
\author{Gabriel Lyra}

\begin{document}
\maketitle



\section{Introdução}

Enquanto a maioria das disciplinas do curso de Ciência da Computação se preocupam com o desenvolvimento de softwares e gerenciamento de dados, a cadeira Interface Usuário Máquina tem como foco principal simplificar a interação do usuário com o produto, tornando-a natural e agradável.  Aprender a projetar tendo em mente não somente a funcionalidade, mas também a acessibilidade e interação do usuário é uma tarefa que muitas vezes é confundida como sendo exclusiva para designers, porém, na realidade é uma questão de suma importância, afinal, um programador deve saber diferenciar quando (e como) programar para pessoas e quando programar para “máquinas”.
 
 
 
\section{Relevância}

São diversas as aplicações dos conhecimentos adquiridos em Interface Usuário Máquina, porém, entre eles, dois se destacam: Design de Interação e qualidade de interface.

No primeiro, nas palavras de Tatiane Cristine, "design de interação significa criar experiências que buscam aperfeiçoar e
estender a maneira como as pessoas trabalham, se comunicam e interagem". Ou seja, a relevância dessa área se encontra na sua consideração pela ergonomia e pelos fatores humanos envolvidos no processo de uso do produto.

No segundo, encontramos a aplicação dos conceitos de Design de Interação. Elementos como responsividade, feedback, estética e perdão então entre os mais importantes no quesito da qualidade (perdão no sentido de que a interface de procurar remediar, senão indicar erros causados pelo usuário). Além disso, podemos destacar que o usuário tende a criar hábitos ao usar a interface fornecida, portanto, também torna-se responsabilidade do designer de garantir a formação de bons hábitos. 




\begin{figure}[h!]
\centering
\includegraphics[width=0.3\textwidth]{Nissan_Leaf_Touchscreen_Auswahlseite.JPG}
\caption{\label{fig:Nissan_Leaf_Touchscreen_Auswahlseite}Exemplo de interface touchscreen implementada em carro, aprimorando a qualidade de acesso à recursos antes indisponiveis ao dirigir-se um carro.(Por J. Hammerschmidt CC-BY-SA 3.0)}
\end{figure}



Em razão dos pontos levantados anteriormente, conclui-se que Interface Usuário Máquina se preocupa em facilitar a interação entre pessoas e produtos. Muito mais do que otimização, refiro-me à capacidade do produto influenciar o usuário a usar o produto de forma mais eficiente, afim de reduzir conflitos entre os dois. Ainda, deve se ter em mente que Interface Usuário Máquina é um tópico extremamente relevante, tanto hoje quanto para o futuro, afinal este topico está diretamente relacionado com aprimoramento e qualidade de vida, estes que são cada vez mais buscados nas indústrias e vidas dos usuários, respectivamente.



\begin{figure}[h!]
\centering
\includegraphics[width=0.3\textwidth]{1280px-Reactable_Multitouch.jpg}
\caption{\label{fig:1280px-Reactable_Multitouch}Avanços de tecnologias já existentes permitem maior interação e compartilhamento entre usuários, a exemplo da interface Multitouch. (CC-BY-SA 2.0)}
\end{figure}



\section{Relação com outras disciplinas}

\begin{table}[h!]
\centering
\begin{tabular}{ | l | p{7.2cm} |}
\hline
Introdução à programação - IF669 & Afim de aplicar os conceitos aprendidos na cadeira de Interface Usuário Máquina o aluno deve estar familiarizado com os conceitos basicos de programação.  \\  \hline

Inglês para computação - LE530 & Ao focar nos usuários deve-se entender que a escolha da linguagem utilizada é essencial, e possivelmente o aluno se deparará com situações em que terá que utilizar o inglês como ferramenta de comunicação ao usuário. \\ \hline

Introdução à multimídia - IF687 & Existem diversas formas de gerar feedbacks a usuário. A cadeira IF687 tem como objetivo apresentar e ensinar os metodos mais diversos, que variam desde o tradicional uso de som, até o recente metódo que é a realidade virtual.\\ \hline

Informação e sociedade - IF679 & Proporcionar a melhor experiência ao usuário é a prioridade, portanto o aluno deve conhecer os conceitos gerais basicos de sociologia, além ser capaz de perceber as percepções e expectativas dos usuários.  \\ \hline

\end{tabular}
\end{table}



\bibliographystyle{alpha}
\bibliography{glgs2}

\cite{designdeinteracao}
\cite{understandingusers}
\cite{designresearch}
\cite{tccalexandre}
\cite{wiki-userinterface}

\end{document}